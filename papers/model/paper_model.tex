%  LaTeX support: latex@mdpi.com 
%  For support, please attach all files needed for compiling as well as the log file, and specify your operating system, LaTeX version, and LaTeX editor.

%=================================================================
\documentclass[gene,journal,article,submit,moreauthors,pdftex]{Definitions/mdpi} 
\usepackage{natbib}
\usepackage[colorinlistoftodos]{todonotes}
\newcounter{mycomment}
\newcommand{\mycomment}[2][]{%
% initials of the author (optional) + note in the margin
\refstepcounter{mycomment}%
{%
\setstretch{0.7}% spacing
\todo[color={red!100!green!33},size=\small]{%
\textbf{Comment [\uppercase{#1}\themycomment]:}~#2}%
}}
% For posting an early version of this manuscript as a preprint, you may use "preprints" as the journal and change "submit" to "accept". The document class line would be, e.g., \documentclass[preprints,article,accept,moreauthors,pdftex]{mdpi}. This is especially recommended for submission to arXiv, where line numbers should be removed before posting. For preprints.org, the editorial staff will make this change immediately prior to posting.

%--------------------
% Class Options:
%--------------------
%----------
% journal
%----------

%---------
% article
%---------
% The default type of manuscript is "article", but can be replaced by: 
% abstract, addendum, article, book, bookreview, briefreport, casereport, comment, commentary, communication, conferenceproceedings, correction, conferencereport, entry, expressionofconcern, extendedabstract, datadescriptor, editorial, essay, erratum, hypothesis, interestingimage, obituary, opinion, projectreport, reply, retraction, review, perspective, protocol, shortnote, studyprotocol, systematicreview, supfile, technicalnote, viewpoint, guidelines, registeredreport, tutorial
% supfile = supplementary materials

%----------
% submit
%----------
% The class option "submit" will be changed to "accept" by the Editorial Office when the paper is accepted. This will only make changes to the frontpage (e.g., the logo of the journal will get visible), the headings, and the copyright information. Also, line numbering will be removed. Journal info and pagination for accepted papers will also be assigned by the Editorial Office.

%------------------
% moreauthors
%------------------
% If there is only one author the class option oneauthor should be used. Otherwise use the class option moreauthors.

%---------
% pdftex
%---------
% The option pdftex is for use with pdfLaTeX. If eps figures are used, remove the option pdftex and use LaTeX and dvi2pdf.

%=================================================================
% MDPI internal commands
\firstpage{1} 
\makeatletter 
\setcounter{page}{\@firstpage} 
\makeatother
\pubvolume{1}
\issuenum{1}
\articlenumber{0}
\pubyear{2021}
\copyrightyear{2020}
%\externaleditor{Academic Editor: Firstname Lastname} % For journal Automation, please change Academic Editor to "Communicated by"
\datereceived{} 
\dateaccepted{} 
\datepublished{} 
\hreflink{https://doi.org/} % If needed use \linebreak
%------------------------------------------------------------------
% The following line should be uncommented if the LaTeX file is uploaded to arXiv.org
%\pdfoutput=1

%=================================================================
% Add packages and commands here. The following packages are loaded in our class file: fontenc, inputenc, calc, indentfirst, fancyhdr, graphicx, epstopdf, lastpage, ifthen, lineno, float, amsmath, setspace, enumitem, mathpazo, booktabs, titlesec, etoolbox, tabto, xcolor, soul, multirow, microtype, tikz, totcount, changepage, paracol, attrib, upgreek, cleveref, amsthm, hyphenat, natbib, hyperref, footmisc, url, geometry, newfloat, caption

%=================================================================
%% Please use the following mathematics environments: Theorem, Lemma, Corollary, Proposition, Characterization, Property, Problem, Example, ExamplesandDefinitions, Hypothesis, Remark, Definition, Notation, Assumption
%% For proofs, please use the proof environment (the amsthm package is loaded by the MDPI class).

%=================================================================
% Full title of the paper (Capitalized)
\Title{A modelling strategy to improve cacao quality and productivity }

% MDPI internal command: Title for citation in the left column
\TitleCitation{Title}

% Author Orchid ID: enter ID or remove command
\newcommand{\orcidauthorA}{0000-0000-0000-000X} % Add \orcidA{} behind the author's name
%\newcommand{\orcidauthorB}{0000-0000-0000-000X} % Add \orcidB{} behind the author's name

% Authors, for the paper (add full first names)
\Author{Angela Romero V $^{1,\dagger,\ddagger}$\orcidA{}, Adriana Gallego $^{1,\ddagger}$ and Anyela V. Camargo R.$^{1,}$*}

% MDPI internal command: Authors, for metadata in PDF
\AuthorNames{Firstname Lastname, Firstname Lastname and Firstname Lastname}

% MDPI internal command: Authors, for citation in the left column
\AuthorCitation{Romero, A.; Gallego, A.; Camargo, A.}
% If this is a Chicago style journal: Lastname, Firstname, Firstname Lastname, and Firstname Lastname.

% Affiliations / Addresses (Add [1] after \address if there is only one affiliation.)
\address{%
$^{1}$ \quad The John Bingham Laboratory, NIAB, 93 Lawrence Weaver Road, Cambridge CB3 0LE, UK 1; e-mail@e-mail.com\\
$^{2}$ \quad BIOS 2; e-mail@e-mail.com}

% Contact information of the corresponding author
\corres{Correspondence:anyela.camargorodriguez@niab.com; Tel.: (optional; include country code; if there are multiple corresponding authors, add author initials) +xx-xxxx-xxx-xxxx (F.L.)}

% Current address and/or shared authorship
\firstnote{Current address: Affiliation 3} 
\secondnote{These authors contributed equally to this work.}
% The commands \thirdnote{} till \eighthnote{} are available for further notes

%\simplesumm{} % Simple summary

%\conference{} % An extended version of a conference paper

% Abstract (Do not insert blank lines, i.e. \\) 
\abstract{Crop modelling can support agronomical decisions of crop production under a range of scenarios improving competitiveness. Cocoa production systems in Latin America has a high importance over social and economic development, facing  the fight against hunger and poverty. Cocoa fruits development can be considered as the result of a number of physiological and morphological processes that can be described by mathematical relationships even under uncontrolled environments. We parametrized the SIMPLE crop model \citep{Zao2019simple} to predict the best time for harvest cocoa fruits in Colombia. The aim of this study is develop a practical tool for farmers using flowering date and weather variables (Solar radiation, rain and temperature) easily available for crop modelling to understand and predict the phenology of cacao trees which is key to produce high quality cacao beans. \textcolor{blue}{need to complete}.  }

% Keywords
\keyword{ICS95; CCN51; thermal time, flowering date } 

%%%%%%%%%%%%%%%%%%%%%%%%%%%%%%%%%%%%%%%%%%
\begin{document}
%%%%%%%%%%%%%%%%%%%%%%%%%%%%%%%%%%%%%%%%%%
%\setcounter{section}{-1} %% Remove this when starting to work on the template.

\section{Introduction}

Cocoa ( \textit{Theobroma cacao }L. ) is an important worldwide perennial tropical crop endemic to the South American rainforests \citep{zuidema2005, motamayor2002, argout2011, Rodriguez2019}. Cacao plant member of the Malvaceae (formerly Sterculiaceae)  botanical family such as  cotton \textit{ Gossypium hirstium} \citep{Nix2017cotton} wich is modeled in SIMPLE model \citep{Zao2019simple}. Cocoa is grown for its fruits, known as cacao pods.  \citep{ Niemenak2010, suarez2021}. Only the 5\% of the world cocoa yield is desalinated for Fine-cocoa production due to the low productivity of  the traditional crop management \citep{argout2011}.  In Colombia, cocoa  is  traditionally  consumed  as  a  beverage. It is one of the crops promoted by the Colombian government in the social and agricultural development  programs aimed at favouring peace in post-conflict regions \citep{Rodriguez2019, Abbott2019} as cocoa is grown by approximately 52.000 \citep{Gutierrez2020} and 98\% of production being carried out by small and medium-sized producers \citep{Garcia2014, Escobar2020}. Colombia registered an increase of 3.750 tons in production in 2020 compared to the previous year \citep{lamos2020}. 

Although Colombian cocoa has the potential to be in the high value markets for fine flavour \citep{Escobar2020}, it is still not widely produced as the lack of adoptions of technologies by the traditional farmers. They empirically harvest after 5 or 6 months after flowering date, hence they ferment cocoa beans without considering the  quality of the seeds at the harvest time. This produce heterogeneous characteristics between each fermentation batch diminishing the quality of cocoa final product \citep{Escobar2021}. To identify the best moment to harvest is important to consider physiological responses affected by climate variables such as rain, solar radiation and wind.  Thus,  for cocoa in Colombia, physiological simulation models may be valuable to identify the best moment to harvest cocoa considering variable weather conditions, soil types and cultivar specifications. 

Crop models represent a quantitative assumption of plant growth depending on sunlight interception efficiency values and climate data supported by a large amount of empirical and ground data. \citep{Reynolds2018}. Physiological crop models have shown to be very useful tool for provided agronomical advices and improvements of the cropping systems of annual crops mainly. Recently crop modelling studies are focusing on  perennial crops  production \citep{zuidema2005, Zao2019simple, Bai2020, Romero2021}. However, the information reported  is be less than for annual crops due to the lack of field data available , relatively high research costs and the difficulties of accumulated errors in long-term simulations \citep{zuidema2005}. For cacao there  approaches  to predict yield mainly  using algorithms of machine learning \citep{lamos2020} and just one mechanistic model simulates physiological cocoa performance  "SUCROS-cocoa" \citep{zuidema2005}. This crop model calculates light interception, photosynthesis, maintenance respiration,
evapotranspiration, biomass production and cocoa yield. It can be parametrised having data on cocoa physiology and morphology \citep{zuidema2005}. However,  there is not specific cocoa physiology data available from small and medium-sized producers. Thus, we adapted the simple generic crop model (SIMPLE) that could be easily modified for any crop to simulate development, crop growth and yield using few parameters such as weather and cultivar specification \citep{Zao2019simple}.

In this paper, we present a physiological parametrization of SIMPLE crop model for cocoa to predict the best harvest time  and yield production. We used the SIMPLE crop model \citep{Zao2019simple} for three reasons: 1: That it is very comprehensively described in the original paper. 2: That the code was available in R for initial trials and 3: That it had already been successfully fitted to perennial crops in south America. Overall, the model simulates crop development, growth and yield, and predict the maturation day when the fruit is ready to harvest. It includes 13 parameters ( daily weather data, irrigation, and soil and key dates) to specify a crop type, with four of these for cultivar characteristics easily available from farmer. Thus, this could be used as  tool for small farmers with the aim to improve the quality of cocoa to become more competitive in Fine-cocoa market.
 
%%%%%%%%%%%%%%%%%%%%%%%%%%%%%%%%%%%%%%%%%%
\section{Materials and Methods}
\subsection{Floral Phenology of Cocoa }
Normally the phenological stages in cocoa are  divided in two main phases: vegetative and reproductive. In the SIMPLE model we simulate reproductive phase described by the floral phenology from the date of inflorescence emergence (BBCH scale 5) to predict the date of ripening of fruit and seed (BBCH 8) \citep{Niemenak2010} (See figure \ref{fig:pheno}). In the Andean region the reproductive phase is cyclically fulfilled during two annual cycles passing by the following phases: inflorescence emergence, flowering, pollination, fruit development and harvest. Therefore, for modelling parametrization the crop cycle of cocoa as perennial plant does not start at the plantation date such as annual crops systems. Instead  the start point of the cocoa crop cycle to model is the inflorescence emergence date. Consequently , the growth period of the fruit can varied from  110 to 150  daa (days after anthesis) \citep{lopez2018} when cacao fruits reaches the physiological maturity, but it can be harvested at 170 days daa \citep{Niemenak2010} for quality purposes.

Cocoa is a cauliflorous plant, which means that flowers grow on the trunk and branches. In Colombia cocoa trees usually produce flowers throughout the year, however the biggest flowering occurs in September and January to harvest in March and July (Fig.\ref{fig:yield}). Cocoa trees produces with up to 10000 flowers per tree each year. which the 50 \%  do not develop into ripe fruits according to Fedecacao reports. The flower takes 30 days passing by 12 micro-stages from meristem development (stages 1–6) to the fully developed flower (Stages 7–12) \citep{swanson2005} when it is ready to be pollinated. 2005). The opening of flowers or anthesis  occurs over a 12-hour period during the night and it is synchronised between the groups of mature flowers \citep{Niemenak2010}. However, the live of a flower can last approximately 1 day after the opening falling form the trunk if it is unfertilised \citep{cheesman1927, Niemenak2010}.

Subsequentially, after anthesis the fruit growths by approximately 150 days until the maturation, mucilage. Therefore, the complete maturation process of the fruit, from the pollination to fully mature fruit, takes 160–210 days \cite{berry1994}. The accumulation of lipids, storage proteins and anthocyanin starts about 85 days after pollination when fruits have an active metabolism and seeds   moisture content decreases up to 30\% \cite{Lehrian1980, Niemenak2010}. During this phase the quality of cocoa seeds is defined.\\
 
 
\begin{figure}[h!]
	\centering
	\caption{\footnotesize {Phenology of cocoa in Colombia for crop modelling.\\}} 
	\includegraphics[scale=0.3]{images/phenology.png}\\
	\footnotesize{Credits: Taken from from Dreamstime.com, phys.org \citep{toledo2021} and \cite{lopez2018}}
	\label{fig:pheno}
\end{figure}


\begin{figure}[h]
	\centering
	\caption{\footnotesize {Average of cocoa production characterization of cocoa for four locations\\}}
	\includegraphics[scale=0.35]{images/yieldmonth.png}\\
	{\footnotesize Flowering occurs in September and January to harvest six months latter in March and July respectively.}
	\label{fig:yield}
\end{figure}
\newpage


\subsection{Test Site}
Cocoa field are located in Saravena (Arauca), Rionegro (Santander), Cali (Valle del Cauca), Apartado (Antioquia) and Manizalez (Caldas). We considered 112 parcels data from Fedecacao reports.Each parcel data contained age of crop, density of planting, yield, number of fruits harvested, flowering date, harvest date. 

\subsection{Input data acquisition for SIMPLE crop model calibration}
Input variables required to run SIMPLE model for cocoa include the flowering date and daily weather of solar radiation (SRAD), maximum and minimum temperature (TMAX, TMIN) and rain. Weather data as csv file was  downloaded from the POWER Data Access Viewer \citep{nasapower} from January 01 2018 to December 31 2020 for four locations in Colombia (Cali, Rionegro -Santander, Apartado - Antioquia, Saravena Arauca). The csv file had to be transformed to .WHT file using R 1.4 version \citep{Rstudio2020}. 

\subsection{Climatological Conditions}
The temperature ranges from 16 to 28 $^\circ$C and it is relatively constant during the year. Saravena presented the maximum variability with hotter months during the first half of the year.  Apartado was found as the hottest with 26 $^\circ$C and Caldas as the coolest site with 18 $^\circ$C compared with other regions tested. Cali and Santander have similar temperature but contrasting PAR conditions. Santander and Caldas have the biggest variability and the maximum of solar radiation values over 20 MJ m$^{2}$day$^{-1}$. In contrast , Cali and Saravena presented the lowest values of PAR below 5 MJ m$^{2}$day$^{-1}$. (Fig.\ref{fig:temp}). Precipitation in Colombia is presented in two seasons per year from February to April and from October to November, hile the relative humidity remain constant (Fig.\ref{fig:rain}).

\begin{figure}[h!]
	\centering
	\caption{\footnotesize {Daily average temperature and available photosynthetic solar radiation (PAR) conditions. \\}}
	\includegraphics[scale=0.25]{images/tempe.png}
	\includegraphics[scale=0.25]{images/SRAD.png}
	\label{fig:temp}
\end{figure}


\begin{figure}[h!]
	\centering
	\caption{\footnotesize {Precipitation and relative humidity per month in Colombia \\}}
	\includegraphics[scale=0.4]{images/rainmonth.png}
	\label{fig:rain}
\end{figure}
\newpage

\subsection{Thermal time for pod harvest date identification}
Thermal time (degree days) was calculated as cumulative sum of temperature from the flowering date to the pod harvest. It is required to model crop growth  considering the cocoa base temperature (T$_{b}$), at which the plant development stops \citep{Ritchie1991}. Thus a cocoa T$_{b}$ of 10$^\circ$C was considered according to \cite{lahive2019}. This characterization was conducted for the five regions counted 180 days (6 months) and 210 days (7 months just for Cladas) after flowering, as farmers used to harvest by calendar days. Characterizing the thermal time the models predicts the maturation day to harvest the cocoa can vary depending on temperature variations.    


\subsection{SIMPLE crop model calibration}

The procedure for the SIMPLE model \citep{Zao2019simple}calibration used was a sequential process of modifying or adding the appropriate files, changing the parameters in the simulation management file, running the program and inspecting the results. This cycle was then repeated until the  simulated yields for cocoa were reasonably close to the observed yield. In the SIMPLE model dummy files are provided for adding new cocoa crop data and weather data, and files 2 to 6 in the list below can be edited to define new cultivars or experiments. Then modifying the simulation management file will cause the new files to be read when the program is run.
\begin{enumerate}
	\item Input/Simulation Management.csv
	\item Input/Species parameter.csv
	\item Input/Cultivar.csv
    \item Input/Treatment.csv	
    \item Input/Irrigation.csv
    \item Input/Soil.csv
    \item Observation/Obsdummy crop Exp name.csv	
    \item Weather/dummy weather.WTH
\end{enumerate}

\subsection{Parameters}
There were three parameters varied by region (table \ref{tab:reparam}) thermal time required for pod harvest after the flowering date (Tsum) , the Radiation use efficiency (RUE) and yield observed on field. Physiological parameters in table \ref{tab:Treaparam} are  common  for all the regions studied. These parameter were calibrate for cultivars ICS95 and CCN51 considering a range of time of 200 days (DAP) from flowering date to harvest day, even thought farmers collect the pod at 180 DAP. Heat and water stress parameters were not considered.


\begin{table}[h!]	
	\caption {\footnotesize {Cocoa crop parameter values used per region.}}
	\label{tab:reparam} 
	\centering
	\begin{small}
		\begin{tabular}{l c c c }
			\hline
			{\bf Cultivar }&{\bf Tsum }&{\bf RUE}&{\bf Yield$^{*}$}\\
			\hline
			Santander & 2016 &0.6 & 2687 \\
			Apartado   & 2906 & 0.6 & 3378  \\
			Arauca   & 1912 & 0.7 & 3981  \\
			Cali   & 2764 & 0.5 & 1900  \\
			Caldas   & 1192 & 0.6 & 740  \\
			\hline
		\end{tabular} \\
		{\footnotesize $^{*}$ Yield observed kg ha$^{-1}$ per year} 
	\end{small}
\end{table}


\begin{table}[h!]	
	\caption {\footnotesize {Parameter values used to run SIMPLEcocoa model.}}
	\centering
     \label{tab:Treaparam} 
	\begin{small}
		\begin{tabular}{{l l l}}
			\hline
            {\bf File }&{\bf Variable name }&{\bf Value}\\
			\hline
			&SoilName & Loamy sand4\\
			&InitialFsolar & 0.01\\
			Treatment&Weather & KOKO (.WTH file name)\\
			&CO$_{2}$ & 400\\
			&SowingDate &Flowering date\\
			\hline
			&Crop cycle DAP & 200 days\\
			&LAI & 1.8 \\
			Observation&FSolar& 0.70\\
			&Biomass & 40kg dry mass per plant\\
			\hline
			&Harvest index & 0.3\\
			Cultivar&150A & 680 $^\circ$C day \\
			&150B & 680 $^\circ$C day \\
			\hline
			&Tbase & 10$^\circ$C\\
			&Topti & 26$^\circ$C \\
			Species&MaxT & 35$^\circ$C \\
			&ExtremeT & 40$^\circ$C  \\
			&CO$_{2}$RUE & 0.09$^\circ$C  \\			
			&S-water & 0 ARID index \\
			\hline			
		\end{tabular} \\ 
	\end{small}
	{\footnotesize S-water is associated drought stress evaluations ranging from 0 (no water shortage) to 1 (extreme water shortage) \cite{Zao2019simple} }
\end{table}



\subsection{Evaluation of model performance}

The SIMPLE model performance was evaluated by comparing simulated values cocoa yield with those reported by Fedecacao from cocoa plantations, using statistical indices of relative-RMSE (RRMSE) (Equ. \ref{equ:RMSE}) \citep{Zao2019simple} and coefficient of determination (R$^{2}$) (Equ. \ref{equ:R2}) \citep{Zao2019simple, Bai2020, Camargo2019aquacropr}. The harvest date predicted was supported by \textcolor{red}{Adriana results}


Yield reports from 20190 to 2020  were shared by Federación Nacional de Cacaoteros Fedecacao. 

\begin{equation}
RMSE= \sqrt{\frac{1}{n}  \sum_{i=1}^{n} (Y_{i}-X_{i})^{2} } 
\label{equ:RMSE}
\end{equation}

\begin{equation}
R^{2}= 1- \frac{\sum_{i=1}^{n} (Y_{i}-X_{i})^{2}}{{\sum_{i=1}^{n}(Y_{i}-0)}^{2}}
\label{equ:R2}
\end{equation}

%%%%%%%%%%%%%%%%%%%%%%%%%%%%%%%%%%%%%%%%%%
\section{Results}

\subsection{Thermal time per region}
The thermal time characterization showed differences per region as was expected. Figure \ref{fig:ttbox} shows degree days ($^\circ$C) of cocoa growth cycles. Caldas had the lowest values and Apartado had the the highest results. this disttribucion is conincident with the temperature (Fig.\ref{fig:temp}). Meanwhile, Cali and Santander presented similar thermal time arround 2016 $^\circ$C.

\begin{figure}[h!]
	\centering
	\caption{\footnotesize {Thermal time characterization of cocoa for five locations.\\}} 
	\includegraphics[scale=0.3]{images/ttbbox.png}
	\label{fig:ttbox}
\end{figure}

\subsection{Weather effects}
Studying the weather data effects over the flowering date and their final yield. In a preliminary analysis of Santanter data, We found that the number of successful flowers pollinated to produce final yield can be affected mostly by the rain, TMAX and wind (Figure \ref{fig:heat}). Farmers 


\begin{figure}[h!]
	\centering
	\caption{\footnotesize {SIMPLE model output for cocoa in Colombia.\\ }} 
	\includegraphics[scale=0.35]{images/outmodel.png}
	\label{fig:m1}
\end{figure}

\subsection{Day of harvest prediction}
\begin{figure}[h!]
	\centering
	\caption{\footnotesize {Day of harvest cocoa simulation\\ }} 
	\includegraphics[scale=0.4]{images/RegionHarvest.png}
	\label{fig:dayH}
\end{figure}
%\begin{figure}[h!]
%	\centering
%	\caption{\footnotesize {Weather variables during flowering times.\\}} 
%	\includegraphics[scale=0.4]{images/averageS.png}
%	\label{fig:weather}
%\end{figure}


\begin{figure}[h!]
	\centering
	\caption{\footnotesize {Pearson correlation among weather variables and flowering dates all locations. \\}} 
	\includegraphics[scale=0.4]{images/heatm.png}\\
	\label{fig:heat}
	{\footnotesize thermalT= thermal time without Tb. ttb= thermal time with Tb. WS2M= Wind Speed at 2 Meters (m/s). T2MDEW = Dew/Frost Point at 2 Meters ($^\circ$C) }
\end{figure}
\newpage

%%%%%%%%%%%%%%%%%%%%%%%%%%%%%%%%%%%%%%%%%%

\section{Discussion}
Authors should discuss the results and how they can be interpreted from the perspective of previous studies and of the working hypotheses. The findings and their implications should be discussed in the broadest context possible. Future research directions may also be highlighted.


The optimum temperature for photosynthesis in cacao has
been reported to be between 31 °C and 33 °C (Balasimha et al.
1991) and 33 °C–35 °C (Yapp 1992). Sena Gomes and
Kozlowski (1987) reported a decrease in gs over the range
18.7 °C to 27.2 °C and an increase at temperatures above this,
most likely to enhance cooling at higher temperatures.
However, the opposite effect was described by Raja Harun
and Hardwick (1988a). In this study, gs and transpiration (E)
increased within the range of 20 °C–30 °C. In field-grown
cacao in India, photosynthetic rate declined as mean monthly
temperature increased above 34 °C during the dry season
(Balasimha et al. 1991). \cite{lahive2019}
\textcolor{red}{this section on going to organize}

Evaluating the level of knowledge of producers regarding cocoa crop management, the harvest was in the group of activities that presented the lowest level of knowledge on the part of the producers according to the general averages \citep{Gutierrez2020}. 

\subsection{Weather Effects}
Wind can affect the availability of tiny flies pollinators from Diptera order and from the families of of the biting midges \textit{Ceratopogonidae},  genus  \textit{Forcipomyia} \citep{Saunders1959, kaufmann1975, sotomayor2020} to reach the cocoa flowers. However, wind is not included in the SIMPLE model as an input as cocoa crop has not been simulated with this model before. Flower opening is very well synchronised between the cohorts of mature flowers opening each night. The flowers open at almost exactly the same time and rate, irrespective of their position on the trunk. Unfertilised flowers abscise from the trunk approximately 1 day after flower opening  \citep{Niemenak2010}. 
%%%%%%%%%%%%%%%%%%%%%%%%%%%%%%%%%%%%%%%%%%
\section{Conclusions}

This section is not mandatory, but can be added to the manuscript if the discussion is unusually long or complex.

%%%%%%%%%%%%%%%%%%%%%%%%%%%%%%%%%%%%%%%%%%


%%%%%%%%%%%%%%%%%%%%%%%%%%%%%%%%%%%%%%%%%%
%% optional
%\supplementary{The following are available online at \linksupplementary{s1}, Figure S1: title, Table S1: title, Video S1: title.}

% Only for the journal Methods and Protocols:
% If you wish to submit a video article, please do so with any other supplementary material.
% \supplementary{The following are available at \linksupplementary{s1}, Figure S1: title, Table S1: title, Video S1: title. A supporting video article is available at doi: link.} 

%%%%%%%%%%%%%%%%%%%%%%%%%%%%%%%%%%%%%%%%%%
\authorcontributions{For research articles with several authors, a short paragraph specifying their individual contributions must be provided. The following statements should be used ``Conceptualization, X.X. and Y.Y.; methodology, X.X.; software, X.X.; validation, X.X., Y.Y. and Z.Z.; formal analysis, X.X.; investigation, X.X.; resources, X.X.; data curation, X.X.; writing---original draft preparation, X.X.; writing---review and editing, X.X.; visualization, X.X.; supervision, X.X.; project administration, X.X.; funding acquisition, Y.Y. All authors have read and agreed to the published version of the manuscript.'', please turn to the  \href{http://img.mdpi.org/data/contributor-role-instruction.pdf}{CRediT taxonomy} for the term explanation. Authorship must be limited to those who have contributed substantially to the work~reported.}

\funding{Please add: ``This research received no external funding'' or ``This research was funded by NAME OF FUNDER grant number XXX.'' and  and ``The APC was funded by XXX''. Check carefully that the details given are accurate and use the standard spelling of funding agency names at \url{https://search.crossref.org/funding}, any errors may affect your future funding.}


\dataavailability{In this section, please provide details regarding where data supporting reported results can be found, including links to publicly archived datasets analyzed or generated during the study. Please refer to suggested Data Availability Statements in section ``MDPI Research Data Policies'' at \url{https://www.mdpi.com/ethics}. You might choose to exclude this statement if the study did not report any data.} 

\acknowledgments{In this section you can acknowledge any support given which is not covered by the author contribution or funding sections. This may include administrative and technical support, or donations in kind (e.g., materials used for experiments).}

\conflictsofinterest{The authors declare no conflict of interest.} 




%% Optional



%%%%%%%%%%%%%%%%%%%%%%%%%%%%%%%%%%%%%%%%%%
\end{paracol}
%%%%%%%%%%%%%%%%%%%%%%%%%%%%%%%%%%%%%%%%%%
% To add notes in main text, please use \endnote{} and un-comment the codes below.
%\begin{adjustwidth}{-5.0cm}{0cm}
%\printendnotes[custom]
%\end{adjustwidth}
%%%%%%%%%%%%%%%%%%%%%%%%%%%%%%%%%%%%%%%%%%
\reftitle{References}

% Please provide either the correct journal abbreviation (e.g. according to the “List of Title Word Abbreviations” http://www.issn.org/services/online-services/access-to-the-ltwa/) or the full name of the journal.
% Citations and References in Supplementary files are permitted provided that they also appear in the reference list here. 

%=====================================
% References, variant A: external bibliography
%=====================================
\externalbibliography{yes}
\bibliography{kokobib}


%=====================================
% References, variant B: internal bibliography
%=====================================
%\begin{thebibliography}{999}
%% Reference 1
%\bibitem[Author1(year)]{ref-journal}
%Author~1, T. The title of the cited article. {\em Journal Abbreviation} {\bf 2008}, {\em 10}, 142--149.
%% Reference 2
%\bibitem[Author2(year)]{ref-book1}
%Author~2, L. The title of the cited contribution. In {\em The Book Title}; Editor1, F., Editor2, A., Eds.; Publishing House: City, Country, 2007; pp. 32--58.
%% Reference 3
%\bibitem[Author3(year)]{ref-book2}
%Author 1, A.; Author 2, B. \textit{Book Title}, 3rd ed.; Publisher: Publisher Location, Country, 2008; pp. 154--196.
%% Reference 4
%\bibitem[Author4(year)]{ref-unpublish}
%Author 1, A.B.; Author 2, C. Title of Unpublished Work. \textit{Abbreviated Journal Name} stage of publication (under review; accepted; in~press).
%% Reference 5
%\bibitem[Author5(year)]{ref-communication}
%Author 1, A.B. (University, City, State, Country); Author 2, C. (Institute, City, State, Country). Personal communication, 2012.
%% Reference 6
%\bibitem[Author6(year)]{ref-proceeding}
%Author 1, A.B.; Author 2, C.D.; Author 3, E.F. Title of Presentation. In Title of the Collected Work (if available), Proceedings of the Name of the Conference, Location of Conference, Country, Date of Conference; Editor 1, Editor 2, Eds. (if available); Publisher: City, Country, Year (if available); Abstract Number (optional), Pagination (optional).
%% Reference 7
%\bibitem[Author7(year)]{ref-thesis}
%Author 1, A.B. Title of Thesis. Level of Thesis, Degree-Granting University, Location of University, Date of Completion.
%% Reference 8
%\bibitem[Author8(year)]{ref-url}
%Title of Site. Available online: URL (accessed on Day Month Year).
%\end{thebibliography}

%\bibliographystyle{te}
%\bibliography{bioinfo}


% If authors have biography, please use the format below
%\section*{Short Biography of Authors}
%\bio
%{\raisebox{-0.35cm}{\includegraphics[width=3.5cm,height=5.3cm,clip,keepaspectratio]{Definitions/author1.pdf}}}
%{\textbf{Firstname Lastname} Biography of first author}
%
%\bio
%{\raisebox{-0.35cm}{\includegraphics[width=3.5cm,height=5.3cm,clip,keepaspectratio]{Definitions/author2.jpg}}}
%{\textbf{Firstname Lastname} Biography of second author}

% The following MDPI journals use author-date citation: Admsci,  Arts, Econometrics, Economies, Genealogy, Humanities, IJFS, Jintelligence, JRFM, Languages, Laws, Literature, Religions, Risks, Social Sciences. For those journals, please follow the formatting guidelines on http://www.mdpi.com/authors/references
% To cite two works by the same author: \citeauthor{ref-journal-1a} (\citeyear{ref-journal-1a}, \citeyear{ref-journal-1b}). This produces: Whittaker (1967, 1975)
% To cite two works by the same author with specific pages: \citeauthor{ref-journal-3a} (\citeyear{ref-journal-3a}, p. 328; \citeyear{ref-journal-3b}, p.475). This produces: Wong (1999, p. 328; 2000, p. 475)

%%%%%%%%%%%%%%%%%%%%%%%%%%%%%%%%%%%%%%%%%%
%% for journal Sci
%\reviewreports{\\
%Reviewer 1 comments and authors’ response\\
%Reviewer 2 comments and authors’ response\\
%Reviewer 3 comments and authors’ response
%}
%%%%%%%%%%%%%%%%%%%%%%%%%%%%%%%%%%%%%%%%%%

\end{document}
